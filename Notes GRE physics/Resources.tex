\section{Recomendations of "Conquering..."}
This is a one-to-one copy of the page xii of the book:\\
\\
Here we collect all the texts we recommend and will refer
to in the review chapters. 

If you're wondering why books by Griffiths show up so often, it's likely because he was on the question-writing committee for the Physics GRE several years ago. Anecdotally, we know that questions are recycled very
often (which is why so few exams have been released), so it's likely that many of the questions you'll see on your exam were written by Griffiths or consciously modeled after his books.

\begin{itemize}
  
\item Classical Mechanics: Whatever book you used for fresh-
man physics should suffice here. For a more in-depth review of advanced topics, try {\bf Classical Dynamics of Particles and Systems by S.T. Thomton and J.B. Marion}.


\item Electricity and Magnetism:  {\bf D.J. Griffiths, Introduction to Electrodynamics}. This book covers everything you'll need to know about electricity and magnetism on the GRE, except
for circuits. For circuits and a review of the most basic
electricity and magnetism problems, which Griffiths glosses over, consult any standard freshman physics textbook. A good treatment of electromagnetic waves can also be found in R.K. Wangsness, Electromagnetic Fields.
E. Purcell, Elec-
tricity and Magnetism is an extremely elegant introduction emphasizing physical concepts rather than mathematical formalism, should you need to relearn the basics of any topic. Under no circumstances should you consult Jackson! It's far too advanced for anything you'll need for the GRE.


\item Optics and Waves: Like classical mechanics, nearly all the relevant information is covered in your freshman physics
textbook. Anything you're missing can be found in the  {\bf relevant chapters of Introduction to Electrodynamics by Griffiths}.


\item Thermodynamics and Statistical Mechanics: No overwhelming recommendation here.  {\bf Thermal Physics and Elementary Statistical Physics by C. Kittel}, or  {\bf Fundamentals of Statistical and Thermal Physics by F. Reif}, are decent. Statistical Physics, by F. Mandl has some decent pedagogy and
the nice feature of many problems with worked solutions. Fermi's Thermodynamics is a classic for the most basic
aspects of the subject.


\item Quantum Mechanics and Atomic Physics:  {\bf  D.J. Griffiths, Introduction to Quantum Mechanics}. This is really the only
reference you need, even for atomic physics questions.  {\bf Shankar and Sakurai are serious overkill, stay away from them for GRE purposes!} 

\item Special Relativity:  {\bf Chapter 12 of Introduction to Electrodynamics by Griffiths}, and  {\bf Chapter 3 of Introduction to Elementary Particles, also by Griffiths}, for more examples of relativistic kinematics. Note that, confusingly, the two books use different sign conventions, so be careful!


\item Laboratory Methods: For advanced circuit elements,  {\bf The Art of Electronics by P. Horowitz and W. Hill is a classic}, and used in many undergraduate laboratory courses. An
excellent general reference for radiation detection is  {\bf Radiation Detection and Measurement by G.F. Knoll. Chapter 1 } covers general properties of radiation, Chapters 2 and 4 cover interactions of radiation with matter, Chapter 10 covers photon detectors, and  {\bf Chapter 3 covers precisely the kind of probability and counting statistics you'll be asked about on the GRE}. The rest of that book goes into far more detail than necessary, so don't worry about it. For lasers, try
O. Svelto, Principles of Lasers, Chapters 1 and 6 .


\item Specialized Topics: The first chapter of  {\bf D.J. Griffiths, Intro-
duction to Elementary Particles, is a mandatory read}.  {\bf It seems that every GRE in the last several years has contained at least one question that can be answered purely by picking facts out of this chapter}. The rest of the book is pretty good too, but the later chapters are almost certainly too advanced for the GRE. 

\item For condensed matter, try  {\bf Introduction to Solid State Physics by C. Kittel}, or Chapters $1-9$ of Solid State Physics by N. Ashcroft and N. Mermin for a more advanced treatment written in a friendly and accessible style.


\end{itemize}

All-around: L. Kirkby's Physics A Student Companion is a
nice all-around summary of a wide range of physics topics.
It's geared toward students studying for exams, so it is
concise and more distilled than the subject-specific books.
There are also several useful websites containing information related to the Physics GRE:
\hyperlink{www.grephysics.net}: A compilation of the 400 problems released by ETS prior to $2011,$ and student-contributed
solutions.


\hyperlink{Link}{www.physicsgre.com} A web forum for discussion of issues related to the GRE, and the grad school application process
in general. Highly recommended: one of us (Y.K.) met sev-
eral future colleagues on this forum before meeting them in
person.


\hyperlink{Link}{www.aps.org/careers/guidance/webinars/gre-strategies.cfm}
A webinar on Physics GRE preparation given by one of us (Y.K.) for the American Physical Society, drawing on strategies discussed in this book.
\section{My special recomendations (Rafael)}