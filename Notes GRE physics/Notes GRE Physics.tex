% LaTeX Notetaking Template for the 2019 Pan-Canadian Self-Organizing Conference on Machine Learning (PC-SOCMLx)
% Template by David Madras, inspired by Meltem Atay's template for the 2018 SOCML

\documentclass[twoside]{scrarticle}
%Descomentar para verla en forma de libro. 
%\documentclass[twoside]{scrbook}
\usepackage[utf8]{inputenc}
\usepackage[english]{babel}
\usepackage{enumitem}
\usepackage{dsfont}
\usepackage{amsmath,amsthm,amssymb}
\usepackage{physics}
\usepackage{xcolor}
\usepackage{lipsum}  
\usepackage[geometry]{ifsym}
\usepackage{hyperref}
\usepackage{float}  

\usepackage[sfdefault]{FiraSans} %% option 'sfdefault' activates Fira Sans as the default text font
\usepackage[T1]{fontenc}
%Descomentar esta linea para comentar la anterior para que slagan los diagramas de feynman compilar con lualatex
%\usepackage{luacode}
%\usepackage[T1]{fontspec}
\usepackage{slashed}
\renewcommand*\oldstylenums[1]{{\firaoldstyle #1}}
\hypersetup{
    colorlinks,
    citecolor=black,
    filecolor=black,
    linkcolor=black,
    urlcolor=black
}
\usepackage{oldgerm}
\usepackage{multicol}
\usepackage{xparse}
\usepackage{environ}

% roman numbering
\renewcommand{\thepart}{\Roman{part}}
% remove dot after number
\renewcommand{\partformat}{\partname~\thepart}
% font for "Part X"
\setkomafont{partnumber}{\normalsize\rmfamily}
% font for part title
\setkomafont{part}{\Large\rmfamily\color{red}\MakeUppercase}

\topmargin -0.2in
\oddsidemargin 0.in
\evensidemargin 0.in
\textheight 8.2in
\textwidth 6.2in
\newtheorem{theorem}{Theorem}[section]
\newtheorem{corollary}{Corollary}[theorem]
\newtheorem{example}{Example}[theorem]
\newtheorem{lemma}[theorem]{Lemma}
\newtheorem{note}[theorem]{Note}
\newtheorem{definition}{Definition}[section]
\renewcommand{\qedsymbol}{$\blacksquare$}

\setlength{\baselineskip}{3.6ex}
\setlength{\unitlength}{.7in}
\parskip 2.ex

\renewcommand{\labelitemi}{$\blacksquare$} 
\title{A brief summary for the GRE-Physics}
\author{Rafael Córdoba L.\\
(And any one whom want to contribute to this text) }
\date{\today}

\begin{document}

\maketitle

This notes will be a friendly remember of physics concepts and formulas ranging from classical mechanics to atomic physics (and beyond if I can).

\begin{itemize}[leftmargin=*,noitemsep]
  \item[] \textbf{Part I of this lecture notes are strongly based in: } Conquering the Physics GRE, Classical mechanics of particles and systems MArion and The Feynman Lectures on Physics Vol I (\hyperlink{Lecture notes heare}{https://www.feynmanlectures.caltech.edu/})
\end{itemize}
\tableofcontents

\part{Classical mechanics}
\section{Statics}
Principal forces involved in free-body diagrams:
\begin{itemize}
    \item Normal force    $\rightarrow$ 3th of Newton betwen the contact surface
    \item Friction $\rightarrow$ $\mu N$
    \item Gravity   mg $\rightarrow$ General form $\frac{Gm_{earth}}{r^2}$
    \item $\dots$
\end{itemize}
\section{Dynamics}
Projectiles:
Aceleration constant. 
$$ma=\underbrace{mg}_{F}\implies \underbrace{\int \int \vec{a} dt}_{\vec{x}} =\int \int \vec{g} dt= \vec{x_0}+\vec{v_0} t+\frac{1}{2}\vec{g}t^2$$
$$\boxed{x_0+v_0t-\frac{1}{2}gt^2=x(t)}$$
By work energy theorem (acceleration constant)
\begin{align*}
\underbrace{\int F \cdot dx }_{\text{Work excerted due the force}}= m\int a dx=ma\Delta x =\underbrace{\frac{1}{2}mv_f^2-\frac{1}{2}mv_i^2}_{\text{Energy needed to lift the projectile}}\\
\end{align*}
$$\boxed{v_f^2-v_i^2 =2a\Delta x}$$
\subsection{Circular motion}
Uniform circular motion:
Radial aceleration creates a {\bf centripetal force} (due to 2nd-Newton)
$$\boxed{a_{\text{radial}}=\frac{v^2}{r}} =\frac{\omega^2}{v}\implies F=\frac{mv^2}{r}$$




\subsection{Energy:}
\begin{itemize}
    \item Traslational $\rightarrow\boxed{ \frac{1}{2}mv^2}$
    \item Rotational   $\rightarrow \boxed{\frac{1}{2}I\omega^2}$
    \item Potential $\rightarrow\boxed{ mgh}$
    \item Spring (Potential) $\rightarrow \boxed{\frac{1}{2}kx^2}$
\end{itemize}

In general $\vec{F}=-\Delta U$ hence, $$\boxed{-\int_\gamma \underbrace{\vec{F}\cdot d^3\vec{l}}_{\text{Path dependent\footnote  {(force perpendicuar to the motion does no work)}}} = \Delta U}$$
\begin{example}{Potential energy of a spring in 3 dimensions}
    $F=-k\vec{x}$ Integrating we find that the energy is:
    $$\Delta U= -\int -k\vec{x} \cdot \underbrace{d^3\vec{x}}_{\vec{n} d^3x}=\int k \begin{bmatrix}
        x &y&z
    \end{bmatrix} \underbrace{\frac{1}{\sqrt{
x^2+y^2+z^2 }}\begin{bmatrix}
        x\\
        y\\
        z\\
    \end{bmatrix}}_{\vec{n}}d^3r =\int k r dr =\frac{1}{2}k r^2=\frac{1}{2}k(x^2+y^2+z^2)$$
\end{example}
\begin{example}
    Energy needed to bring a satellite to a distance r of the earth.

    $$U(r)=-\int \vec{F} \cdot d^3\vec{l}=-\int \underbrace{-\frac{GM_{\text{Earth}} m_{\text{satellite}}}{r^2}}_{\text{Attractive} \mapsto \text{sign negative}}} dr=-\frac{GM_{\text{Earth}} m_{\text{satellite}}}{r}$$
\end{example}

\subsubsection{Some tips to find moments of Inertia (and not have memorize)}

$$\boxed{I= \int r^2 dm}$$

Parallel axis theorem: $$\boxed{I=I_{CM}+Mr^2}
$$
    where r is the distance from the center to the parallel axis  
 
    
{\bf The grater moment inertia the harder to rotate}




In addition, many problems relate rolling with linear velocity {\bf rolling with out slipping} $\boxed{v=R\omega}$ 


\subsection{Collisions}
Remember $\boxed{\vec{p}=m\vec{v} \rightarrow \vec{F}= \dot{\vec{p}}}$ equivalent $\boxed{\vec{L}=\vec{r}\cross \vec{p}=I\vec{\omega} \rightarrow \tau = \dot{\vec{L}}=\vec{r}\cross \vec{F}}$


Evry collison has to have either momentum (Angular or linear) or energy conservation.


Reference frame rotating at a velocity $\vec{
   \Omega 
}$ is not intertial (tangential aceleration) hence, $F=ma$ remains constant if and only if we introduce the folowwing forces:
\boxed{
\begin{aligned}
    F_{\text {centrifugal }} &=-m \Omega^{2} r \\
    F_{\text {Coriolis }} &=-2 m \mathbf{\Omega} \times \mathbf{v}
    \end{aligned}}


    $$\boxed{\mathbf{r}_{\mathrm{CM}}=\frac{\int \mathbf{r} d m}{M}}$$



    \subsection{Lagrangian and Hamiltonian mechanics:}
    The famous and not so famous E-L eq and Hamiltonian eq are:
    $$\boxed{\frac{d}{d t} \frac{\partial L}{\partial \dot{q}}=\frac{\partial L}{\partial q} \quad \dot{p}=-\frac{\partial H}{\partial q}, \quad \dot{q}=\frac{\partial H}{\partial p}}$$
where $\boxed{H(p, q)=\sum_{i} p_{i} \dot{q}_{i}-L}$ and $ p_{i} \equiv \frac{\partial L}{\partial \dot{q}}: \text { momentum conjugate to } q$

Some times the Lagrangian can be express in termos of a new effective potential in such a way that the kinetic energy behaves as isolated coordinates that is, the kinetic energy is of the form $\frac{1}{2}mr^2$.

\subsubsection{Orbits}

\begin{itemize}
    \item E>0 hiperbolic (why?)
    \item E<0 elliptic
    \item E=0 parabolic
\end{itemize}

\subsubsection{Kepler law's}
\begin{itemize}
    \item Planets move in elliptical orbits with one focus at the Sun.
    \item Planetary orbits sweep out equal areas in equal times.
    \item If $T$ is the period of a planetary orbit, and $a$ is the semimajor axis of the orbit, then $T=k a^{3 / 2},$ with $k$ the same
    constant for all planets.
\end{itemize}


\section{Harmonic oscilator}

$$\boxed{
   \omega=\sqrt{
      \frac{k}{m} 
   } 
}$$

\subsubsection{Normal modes}
In general we have to have a system that writes the equation $$\sum_{k}\left(A_{j k} q_{k}+m_{j k} \ddot{q}_{k}\right)=0$$ using the anzats of solution we find the following system of equations $$\sum_{k}\left(A_{j k}-\omega^{2} m_{j k}\right) a_{k}=0$$

Whose solutions or {\bf normal modes} can de determinde by $$\boxed{\operatorname{det}\left(A_{j k}-\omega^{2} m_{j k}\right)=0}$$

Lets not forget, in presence of damping (force proportional to $bv$) we have a form $e^i(sdfsdf+fsdf) $

If the system is driven with a harmonic force (sin or cosine) the force equation gives:
$$\ddot{x}+2 \beta \dot{x}+\omega_{0}^{2} x=A \cos \omega t$$ hence, the resonant frecuencies are $$\boxed{
    \omega_{R}=\sqrt{\omega_{0}^{2}-2 \beta^{2}}
}$$



\section{Fluid mechanics}

$$\boxed{
    \frac{v^{2}}{2}+g z+\frac{p}{\rho}=\mathrm{constant}
    
    \quad F_{
       \text{
          buoyant 
       } 
    }=\rho V g
}$$
\part{Electricity and magnetism}
In the case of electromagnetism the Maxwell equations are the dean of any process(Elecrtomagnetic).
$$\boxed{\begin{aligned}
    \nabla \cdot \mathbf{E} =\frac{\rho}{\varepsilon_{0}}  &\rightarrow  \oint \vec{E} \cdot d \vec{a}=\frac{Q_{e n c}}{\varepsilon_{0}} \\
    \nabla \cdot \mathbf{B} =0  &\rightarrow \oint \vec{B} \cdot d \vec{a}=0 \\
    \nabla \times \mathbf{E} =-\frac{\partial \mathbf{B}}{\partial t}  &\rightarrow  \oint \vec{E} \cdot d \vec{l}=-\int \frac{\partial \vec{B}}{\partial t} \cdot d \vec{a} \\
    \nabla \times \mathbf{B} =\mu_{0} \mathbf{j}+\frac{1}{c^{2}} \frac{\partial \mathbf{E}}{\partial t} &\rightarrow  \oint \vec{B} \cdot \overrightarrow{d l}=\mu_{0} I_{e n c}+\mu_{0} \varepsilon_{0} \int \frac{\partial \vec{E}}{\partial t}
    \end{aligned}}$$
    
\section{Statics (Electric charges)}
In general statics means process that does not vary in time or equivalent mathematically all time derivatives vanishes i.e. $$\frac{\partial \text{Some quantity}}{\partial t} \rightarrow 0$$
Electrostatics follows from the first two Maxwell equations $$\boxed{\Delta \cdot \vec{E}=\frac{\rho}{\varepsilon_0} \quad \Delta \cross \vec{E}=0} $$ 
As gravitational fields electric fields excerts a force given by $$\boxed{\vec{F}_E =q \vec{E}}$$.
Potentials on the other hand are more useful than fields because they are functions(as dummy as it sounds fields are fields).
$$\boxed{\vec{E}=-\Delta V}$$ using the fundamental theorem, $$V(b)=-\int_{\underbrace{a}_{\text{Zero of the potential}}}^{b} \mathbf{E} \cdot d \mathbf{l}$$ Dont forget the sign!
$$\nabla \cdot \mathbf{E} =-{\Delta} V
\rightarrow \Delta V= -\frac{\rho}{\varepsilon_0}$$


To find solutions we recall the Laplace and Poisson differential equations: $\Delta F=g$ (Poisson PED) and $\Delta F= 0$ (Laplace PDE) using green functions (I have some notes if you want to remember Green methods to solve PDE).

$$\boxed{
    $$V(\mathbf{r})=\frac{1}{4 \pi \epsilon_{0}} \int \frac{\rho\left(\mathbf{r}^{\prime}\right)}{\left|\mathbf{r}-\mathbf{r}^{\prime}\right|} d^{3} \mathbf{r}^{\prime}$$ 
}$$

Coulomb's law follows taking $\rho =q \delta^3(\mathbf{x})$ or by symmetr using the first Maxwell eq in the integral form (symmetry tell us the direction of the field in this case $\hat{r}$)

\subsection{A comparation between mechanics and electromagnetism}
$$\begin{align*}
    -4 \pi G \rho =\nabla \cdot \mathbf{g}&\rightarrow \nabla \cdot \mathbf{E} = \frac{\rho}{\varepsilon}\\
    \frac{G m_1\hat{r}}{r^2}= \mathbf{g} &\rightarrow\mathbf{E}=\frac{q_1 \hat{r}}{4\pi \varepsilon r^2}\\
    \frac{G m_1}{r}=U_g \quad -\nabla U_g =\mathbf{g}  &\rightarrow \vec{E}=-\nabla V \quad  V=\frac{q_1 }{4\pi \varepsilon r}\\
    \vec{F}_G = -\nabla U_g & \rightarrow \vec{F}_E =qE=-q\nabla\underbrace{ V}_{
       \text{
          potential energy per unit charge 
       } 
    }
\end{align*}$$



\subsection{What are poles? (Dipoles, octapoles, etc.)}
In principle the dipole are a pair of charges of different sign separated at a distance 2d locatted in the x axis. {\bf ¿Whats the importance of dipoles? I dont know.} At a point $(x,0,0)$ is easy to see that $$V(x)=\frac{1}{4 \pi \epsilon_{0}}\left(\frac{q}{|x-d|}-\frac{q}{|x+d|}\right)$$ 

In the same fashion of dipole, a n-pole is just a combination of symmetry of charges alternating signs. that is, a n-pole forms a n-gon where charges are located in the vertices having alternating signs between neighbors.
\subsection{Planes}
Planes on the other hand are the second most important example (besides cylinders) of the Gauss law. 
$$\oint \mathbf{E}\cdot d\mathbf{A}=E2A=\frac{Q}{\varepsilon_0}=\frac{\sigma A}{\varepsilon_0}$$
Hence, the electirc field of a plane  laying on the xy plane with $\sigma$ positive (postive charged plane) is:
$$\boxed{\mathbf{E}=\frac{\sigma}{2 \epsilon_{0}} \hat{z}}$$
\subsection{Line charges and cylinders}
Similarly as above for a cylinder (gaussian surface of radius $a$) and line (charge) of length $l$, we have: $$\oint \mathbf{E} \cdot d\mathbf{A}=E(2\pi a l)=\frac{Q}{\varepsilon_0}=\frac{\lambda l}{\varepsilon_0}$$

$$\boxed{\mathbf{E}=\frac{\lambda}{2\pi \varepsilon_0 a}\hat{
   a 
}}
$$

\subsection{Boundaries}
At a boundary of a surface we hacethe tangent and normal electric field since tangent is a hyperplane (or a plane ) the normal electrc field is a scalar.

Integrating over a path; specifically a rectangular path with long sides paralel to the boundary and the short ones normal to the boundary (I need graphs here) we apply the second Maxwell law (statics) then $\oint \mathbf{E} \cdot d\mathbf{l}=0$ and let the perpendicual lengths tend to zero:
$$\boxed{\mathbf{E}_{\text {out }}^{\|}-\mathbf{E}_{\mathrm{in}}^{\|}=0}$$
Now, 
$$E_{\mathrm{out}}^{\perp}-E_{\mathrm{in}}^{\perp}=\frac{\sigma}{\epsilon_{0}}$$
Hence, a discontinuty of the electric field is present. in conclusion $$\boxed{
   \text{
      Derivatives of V are continuous, except at charged surfaces  
   } 
}$$
\section{Conductors}
$V$ is constant throughout a conductor which implies the following:
\begin{itemize}
    \item Electric field inside a conductor is zero (derivaitve vanishes)
    \item The net charge density inside a conductor is zero (charge can flow freely)
    \item Any net charge on a conductor is confined to the surface (charges inside cancelout them selves but in the boundaries there is no charge to cancel)
    \item The electric field just  outside a conductor is perpendicular to the surface ()
\end{itemize}
\section{The image method (I finally undestood)}
Since Lapalce equation for the potential $V$ is a linear PDE (and by mathematical artifices) then it has a {
   \bf unique solution 
}  hence if we propose an anzats for $V$ and if turns out that our guess satisfy the conditions (boundaries or charges) then it must be the potential $V$; the image method propose an anzats.


The image method follows from our colloraries above, imagine you have a grounded  (groundend means $V=0$) conducting (conducting means $V=cte$) plane and put a charge $q$ at a distance $d$, in principle ther is 
no symmetry in this problem but from our colloraries we know that $V$ has to vanish in the plane hence,  we propose a imaginary (or virtual) charge called the image charge in such a way that the potential vanishes in the plane that is 

$$V=\frac{q}{4\pi \varepsilon_0 d}+\text{
   somthing 
}=0 $$ then this something is a oposite charge particle with distance d on the other side(on the other side because it has to be {inaxesible
 for the other charge
}) of the plane.

Hence, the field of a plane and a particle with charge q is equivalent to a dipole. It is important to say that $\boxed{\text{
   There is no energy cost to moving an image charge 
}}$ that is because the image charge is equivalent of the charges in the plane and charges on a grounded conucting plane does not uses energy when moving charges (charges obtain is energy from the grounded state).


\section{Work and energy.}
As we said $V$ is the potential energy per unit charge hence, the work done for a system of particles is:
$$\boxed{W=\frac{1}{2} \underbrace{\sum_{i=1}^{n} q_{i} V\left(\mathbf{r}_{i}\right)}_{
   \text{
      Double counts hence devide by 2. 
   } 
}} \rightarrow W=\frac{1}{2} \int \rho(\mathbf{r}) V(\mathbf{r}) d^{3} \mathbf{r}$$
Field theorist may remember that particles carry energy but also fields carry energy by itself. $$\boxed{U_{E}=\frac{\epsilon_{0}}{2} \int|\mathbf{E}|^{2} d^{3} \mathbf{r} \quad U_{B}=\frac{1}{2 \mu_{0}} \int|\mathbf{B}|^{2} d^{3} \mathbf{r}}$$

\subsection{Capaciors}
Suppose you have to conductors (it doesn have to be the same) one charged with $Q$ and the other one with $-Q$. This conductors allow electric fields between them and hence, different potentials call it $0$ and $V$ (They are conductors).
This is a capacitor. $$\boxed{Q=CV}$$ the most important thing is that in many ways C changes with the geometry of the problem.

\begin{example}
    For two planes with $Q$ and $-Q$ the electric field outside is 0 but inside is $$E=\frac{\sigma}{2\varepsilon_0}+\frac{\sigma}{2\varepsilon_0}=\frac{\sigma}{\varepsilon_0}$$ using $V=-\int_0^d \mathbf{E}\cdot d\mathbf{l}=Ed=\frac{\sigma d}{\varepsilon_0}$

    The the capacitance is $$C=\frac{Q}{V}=\frac{\sigma A}{E d}=\frac{\sigma A \varepsilon_0}{\sigma d}=\boxed{\frac{\varepsilon_0 A}{d}=C}$$

    The energy is therfore, $$U_C=\frac{\epsilon_{0}}{2} \int|\mathbf{E}|^{2} d^{3} \mathbf{r}=\frac{\epsilon_{0}}{2} \qty(\frac{Q^2}{\varepsilon^2_0 A^2})\int d^{3} \mathbf{r}=\boxed{\frac{1}{2}\frac{Q^2}{C}=\frac{1}{2}CV^2=U_C}$$
\end{example}

\section{Statics (Magnetiostatics)}
In the same fashion as electrostatics  we use the Maxwell equation with vanishsing time derivative that is :

$$\boxed{\begin{array}{c}
    \nabla \cdot \mathbf{B}=0  \rightarrow  \oint_{S} \mathbf{B} \cdot d \mathbf{S}=0 \\ \\
    \nabla \times \mathbf{B}=\mu_{0} \mathbf{J} \rightarrow \underbrace{\oint_{C} \mathbf{B} \cdot d \mathbf{l}=\mu_{0} I_{\mathrm{enc}}}_{\text {Ampère's law}}
    \end{array}}$$
whose potential is $$\boexed{\nabla \times \mathbf{A}=\mathbf{B}}$$ analogous, the {\bf Lorentz force} is $$\boexed{\mathbf{F}_{B}=q \mathbf{v} \times \mathbf{B}}$$
and for a wire $$\boxed{d \mathbf{F}_{B}=I d \mathbf{l} \times \mathbf{B}}$$

Some useful right-hand results $$\boxed{\hat{\mathbf{z}} \times \hat{\mathbf{r}}=\hat{\boldsymbol{\phi}} \quad \hat{\boldsymbol{\theta}} \times \hat{\boldsymbol{\phi}}=\hat{\mathbf{r}}}$$ are the right hand in spherical and cylindrical coordiantes
\subsection{Ampère's law and Biot-Savart law}
Ampére law is the (functional) analogous of Gauss law in charges. We invoke the use of symmetry: 
\begin{aligned}
    &\mathbf{B} \mid \oint_{C} d l=\mu_{0} I_{\mathrm{enc}}\\
    &|\mathbf{B}|=\frac{\mu_{0} I_{\mathrm{enc}}}{L}
    \end{aligned}
    \begin{example}
        For a wirwe with current I, $$\boxed{|\mathbf{B}|(2 \pi r)=\mu_{0} I \Longrightarrow \mathbf{B}=\frac{\mu_{0} I}{2 \pi r} \hat{\boldsymbol{\phi}}}$$
    \end{example}
    \begin{example}
        For selenoid, the left hand side (LHS) is $BL$(contributions only inside the sleneoid) and hence $$\boxed{BL=\mu_0InL}$$
    \end{example}
    \begin{example}
        For a toroid, we use a circular loop passing the wires inside, hence $$\boxed{B=\frac{\mu_{0} N I}{2 \pi r} \quad(\text { toroid })}$$
    \end{example}
    Many times we do not have symmetry hence, we use Biot -Savart (solutions of the Maxwell equation)
    $$\boxed{\mathbf{B}(\mathbf{r})=\frac{\mu_{0} I}{4 \pi} \int \frac{d \mathbf{l} \times \hat{\mathbf{r}^{\prime}}}{r^{\prime 2}}}$$
\subsection{Boundary conditions}
\subsection{Cyclotron motion}
\section{Dynamics}

\part{Appendix of formulas worth to remember}
\section{Must know formulas}
\section{Interesting but not mandatory}
\section{Specialized formulas}
\section{You only use once in 10000 years formulas}
\part{Resources recomended by "Conquering the GRE..."}
\section{Recomendations of "Conquering..."}
This is a one-to-one copy of the page xii of the book:\\
\\
Here we collect all the texts we recommend and will refer
to in the review chapters. 

If you're wondering why books by Griffiths show up so often, it's likely because he was on the question-writing committee for the Physics GRE several years ago. Anecdotally, we know that questions are recycled very
often (which is why so few exams have been released), so it's likely that many of the questions you'll see on your exam were written by Griffiths or consciously modeled after his books.

\begin{itemize}
  
\item Classical Mechanics: Whatever book you used for fresh-
man physics should suffice here. For a more in-depth review of advanced topics, try {\bf Classical Dynamics of Particles and Systems by S.T. Thomton and J.B. Marion}.


\item Electricity and Magnetism:  {\bf D.J. Griffiths, Introduction to Electrodynamics}. This book covers everything you'll need to know about electricity and magnetism on the GRE, except
for circuits. For circuits and a review of the most basic
electricity and magnetism problems, which Griffiths glosses over, consult any standard freshman physics textbook. A good treatment of electromagnetic waves can also be found in R.K. Wangsness, Electromagnetic Fields.
E. Purcell, Elec-
tricity and Magnetism is an extremely elegant introduction emphasizing physical concepts rather than mathematical formalism, should you need to relearn the basics of any topic. Under no circumstances should you consult Jackson! It's far too advanced for anything you'll need for the GRE.


\item Optics and Waves: Like classical mechanics, nearly all the relevant information is covered in your freshman physics
textbook. Anything you're missing can be found in the  {\bf relevant chapters of Introduction to Electrodynamics by Griffiths}.


\item Thermodynamics and Statistical Mechanics: No overwhelming recommendation here.  {\bf Thermal Physics and Elementary Statistical Physics by C. Kittel}, or  {\bf Fundamentals of Statistical and Thermal Physics by F. Reif}, are decent. Statistical Physics, by F. Mandl has some decent pedagogy and
the nice feature of many problems with worked solutions. Fermi's Thermodynamics is a classic for the most basic
aspects of the subject.


\item Quantum Mechanics and Atomic Physics:  {\bf  D.J. Griffiths, Introduction to Quantum Mechanics}. This is really the only
reference you need, even for atomic physics questions.  {\bf Shankar and Sakurai are serious overkill, stay away from them for GRE purposes!} 

\item Special Relativity:  {\bf Chapter 12 of Introduction to Electrodynamics by Griffiths}, and  {\bf Chapter 3 of Introduction to Elementary Particles, also by Griffiths}, for more examples of relativistic kinematics. Note that, confusingly, the two books use different sign conventions, so be careful!


\item Laboratory Methods: For advanced circuit elements,  {\bf The Art of Electronics by P. Horowitz and W. Hill is a classic}, and used in many undergraduate laboratory courses. An
excellent general reference for radiation detection is  {\bf Radiation Detection and Measurement by G.F. Knoll. Chapter 1 } covers general properties of radiation, Chapters 2 and 4 cover interactions of radiation with matter, Chapter 10 covers photon detectors, and  {\bf Chapter 3 covers precisely the kind of probability and counting statistics you'll be asked about on the GRE}. The rest of that book goes into far more detail than necessary, so don't worry about it. For lasers, try
O. Svelto, Principles of Lasers, Chapters 1 and 6 .


\item Specialized Topics: The first chapter of  {\bf D.J. Griffiths, Intro-
duction to Elementary Particles, is a mandatory read}.  {\bf It seems that every GRE in the last several years has contained at least one question that can be answered purely by picking facts out of this chapter}. The rest of the book is pretty good too, but the later chapters are almost certainly too advanced for the GRE. 

\item For condensed matter, try  {\bf Introduction to Solid State Physics by C. Kittel}, or Chapters $1-9$ of Solid State Physics by N. Ashcroft and N. Mermin for a more advanced treatment written in a friendly and accessible style.


\end{itemize}

All-around: L. Kirkby's Physics A Student Companion is a
nice all-around summary of a wide range of physics topics.
It's geared toward students studying for exams, so it is
concise and more distilled than the subject-specific books.
There are also several useful websites containing information related to the Physics GRE:
\hyperlink{www.grephysics.net}: A compilation of the 400 problems released by ETS prior to $2011,$ and student-contributed
solutions.


\hyperlink{Link}{www.physicsgre.com} A web forum for discussion of issues related to the GRE, and the grad school application process
in general. Highly recommended: one of us (Y.K.) met sev-
eral future colleagues on this forum before meeting them in
person.


\hyperlink{Link}{www.aps.org/careers/guidance/webinars/gre-strategies.cfm}
A webinar on Physics GRE preparation given by one of us (Y.K.) for the American Physical Society, drawing on strategies discussed in this book.
\section{My special recomendations (Rafael)}
\end{document}
