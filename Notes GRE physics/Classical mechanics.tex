\section{Statics}
Principal forces involved in free-body diagrams:
\begin{itemize}
    \item Normal force    $\rightarrow$ 3th of Newton betwen the contact surface
    \item Friction $\rightarrow$ $\mu N$
    \item Gravity   mg $\rightarrow$ General form $\frac{Gm_{earth}}{r^2}$
    \item $\dots$
\end{itemize}
\section{Dynamics}
Projectiles:
Aceleration constant. 
$$ma=\underbrace{mg}_{F}\implies \underbrace{\int \int \vec{a} dt}_{\vec{x}} =\int \int \vec{g} dt= \vec{x_0}+\vec{v_0} t+\frac{1}{2}\vec{g}t^2$$
$$\boxed{x_0+v_0t-\frac{1}{2}gt^2=x(t)}$$
By work energy theorem (acceleration constant)
\begin{align*}
\underbrace{\int F \cdot dx }_{\text{Work excerted due the force}}= m\int a dx=ma\Delta x =\underbrace{\frac{1}{2}mv_f^2-\frac{1}{2}mv_i^2}_{\text{Energy needed to lift the projectile}}\\
\end{align*}
$$\boxed{v_f^2-v_i^2 =2a\Delta x}$$
\subsection{Circular motion}
Uniform circular motion:
Radial aceleration creates a {\bf centripetal force} (due to 2nd-Newton)
$$\boxed{a_{\text{radial}}=\frac{v^2}{r}} =\frac{\omega^2}{v}\implies F=\frac{mv^2}{r}$$




\subsection{Energy:}
\begin{itemize}
    \item Traslational $\rightarrow\boxed{ \frac{1}{2}mv^2}$
    \item Rotational   $\rightarrow \boxed{\frac{1}{2}I\omega^2}$
    \item Potential $\rightarrow\boxed{ mgh}$
    \item Spring (Potential) $\rightarrow \boxed{\frac{1}{2}kx^2}$
\end{itemize}

In general $\vec{F}=-\Delta U$ hence, $$\boxed{-\int_\gamma \underbrace{\vec{F}\cdot d^3\vec{l}}_{\text{Path dependent\footnote  {(force perpendicuar to the motion does no work)}}} = \Delta U}$$
\begin{example}{Potential energy of a spring in 3 dimensions}
    $F=-k\vec{x}$ Integrating we find that the energy is:
    $$\Delta U= -\int -k\vec{x} \cdot \underbrace{d^3\vec{x}}_{\vec{n} d^3x}=\int k \begin{bmatrix}
        x &y&z
    \end{bmatrix} \underbrace{\frac{1}{\sqrt{
x^2+y^2+z^2 }}\begin{bmatrix}
        x\\
        y\\
        z\\
    \end{bmatrix}}_{\vec{n}}d^3r =\int k r dr =\frac{1}{2}k r^2=\frac{1}{2}k(x^2+y^2+z^2)$$
\end{example}
\begin{example}
    Energy needed to bring a satellite to a distance r of the earth.

    $$U(r)=-\int \vec{F} \cdot d^3\vec{l}=-\int \underbrace{-\frac{GM_{\text{Earth}} m_{\text{satellite}}}{r^2}}_{\text{Attractive} \mapsto \text{sign negative}}} dr=-\frac{GM_{\text{Earth}} m_{\text{satellite}}}{r}$$
\end{example}

\subsubsection{Some tips to find moments of Inertia (and not have memorize)}

$$\boxed{I= \int r^2 dm}$$

Parallel axis theorem: $$\boxed{I=I_{CM}+Mr^2}
$$
    where r is the distance from the center to the parallel axis  
 
    
{\bf The grater moment inertia the harder to rotate}




In addition, many problems relate rolling with linear velocity {\bf rolling with out slipping} $\boxed{v=R\omega}$ 


\subsection{Collisions}
Remember $\boxed{\vec{p}=m\vec{v} \rightarrow \vec{F}= \dot{\vec{p}}}$ equivalent $\boxed{\vec{L}=\vec{r}\cross \vec{p}=I\vec{\omega} \rightarrow \tau = \dot{\vec{L}}=\vec{r}\cross \vec{F}}$


Evry collison has to have either momentum (Angular or linear) or energy conservation.


Reference frame rotating at a velocity $\vec{
   \Omega 
}$ is not intertial (tangential aceleration) hence, $F=ma$ remains constant if and only if we introduce the folowwing forces:
\boxed{
\begin{aligned}
    F_{\text {centrifugal }} &=-m \Omega^{2} r \\
    F_{\text {Coriolis }} &=-2 m \mathbf{\Omega} \times \mathbf{v}
    \end{aligned}}


    $$\boxed{\mathbf{r}_{\mathrm{CM}}=\frac{\int \mathbf{r} d m}{M}}$$



    \subsection{Lagrangian and Hamiltonian mechanics:}
    The famous and not so famous E-L eq and Hamiltonian eq are:
    $$\boxed{\frac{d}{d t} \frac{\partial L}{\partial \dot{q}}=\frac{\partial L}{\partial q} \quad \dot{p}=-\frac{\partial H}{\partial q}, \quad \dot{q}=\frac{\partial H}{\partial p}}$$
where $\boxed{H(p, q)=\sum_{i} p_{i} \dot{q}_{i}-L}$ and $ p_{i} \equiv \frac{\partial L}{\partial \dot{q}}: \text { momentum conjugate to } q$

Some times the Lagrangian can be express in termos of a new effective potential in such a way that the kinetic energy behaves as isolated coordinates that is, the kinetic energy is of the form $\frac{1}{2}mr^2$.

\subsubsection{Orbits}

\begin{itemize}
    \item E>0 hiperbolic (why?)
    \item E<0 elliptic
    \item E=0 parabolic
\end{itemize}

\subsubsection{Kepler law's}
\begin{itemize}
    \item Planets move in elliptical orbits with one focus at the Sun.
    \item Planetary orbits sweep out equal areas in equal times.
    \item If $T$ is the period of a planetary orbit, and $a$ is the semimajor axis of the orbit, then $T=k a^{3 / 2},$ with $k$ the same
    constant for all planets.
\end{itemize}


\section{Harmonic oscilator}

$$\boxed{
   \omega=\sqrt{
      \frac{k}{m} 
   } 
}$$

\subsubsection{Normal modes}
In general we have to have a system that writes the equation $$\sum_{k}\left(A_{j k} q_{k}+m_{j k} \ddot{q}_{k}\right)=0$$ using the anzats of solution we find the following system of equations $$\sum_{k}\left(A_{j k}-\omega^{2} m_{j k}\right) a_{k}=0$$

Whose solutions or {\bf normal modes} can de determinde by $$\boxed{\operatorname{det}\left(A_{j k}-\omega^{2} m_{j k}\right)=0}$$

Lets not forget, in presence of damping (force proportional to $bv$) we have a form $e^i(sdfsdf+fsdf) $

If the system is driven with a harmonic force (sin or cosine) the force equation gives:
$$\ddot{x}+2 \beta \dot{x}+\omega_{0}^{2} x=A \cos \omega t$$ hence, the resonant frecuencies are $$\boxed{
    \omega_{R}=\sqrt{\omega_{0}^{2}-2 \beta^{2}}
}$$



\section{Fluid mechanics}

$$\boxed{
    \frac{v^{2}}{2}+g z+\frac{p}{\rho}=\mathrm{constant}
    
    \quad F_{
       \text{
          buoyant 
       } 
    }=\rho V g
}$$